%% Bunch of commonly used macros

\usepackage{wrapfig}

\newcommand{\gke}{\texttt{Gkeyll}}

\newcommand{\attrib}[1]{
\nopagebreak{\raggedleft\footnotesize #1\par}\vskip0.1in}
\renewcommand{\poemtitlefont}{\normalfont\large\itshape\centering}

\newtheorem{proposition}{Proposition}
\newtheorem{theorem}{Theorem}
\newtheorem{lemma}{Lemma}
\newtheorem{remark}{Remark}
\newtheorem{definition}{Definition}
\newtheorem{principle}{Principle}
\newtheorem{identity}{Identity}
\theoremstyle{definition}
\newtheorem{exmp}{Example}

\theoremstyle{definition}
\newtheorem{entry}{Entry}

\theoremstyle{definition}
\newtheorem{exer}{Exercise}

\DeclareMathAlphabet{\mathpzc}{OT1}{pzc}{m}{it}

% auto-scaled figured
\newcommand{\incfig}{\centering\includegraphics}
\setkeys{Gin}{width=0.9\linewidth,keepaspectratio}

% Commonly used macros
\newcommand{\eqr}[1]{Eq.\thinspace(#1)}
\newcommand{\pfrac}[2]{\frac{\partial #1}{\partial #2}}
\newcommand{\pfracc}[2]{\frac{\partial^2 #1}{\partial #2^2}}
\newcommand{\pfraca}[1]{\frac{\partial}{\partial #1}}
\newcommand{\pfracb}[2]{\partial #1/\partial #2}
\newcommand{\pfracbb}[2]{\partial^2 #1/\partial #2^2}
\newcommand{\spfrac}[2]{{\partial_{#1}} {#2}}
\newcommand{\mvec}[1]{\mathbf{#1}}
\newcommand{\bmvec}[1]{\breve{\mathbf{#1}}}
\newcommand{\gvec}[1]{\boldsymbol{#1}}
\newcommand{\script}[1]{\mathpzc{#1}}
\newcommand{\gcs}{\nabla_{\mvec{x}}}
\newcommand{\gvs}{\nabla_{\mvec{v}}}

\newcommand{\nGA}[1]{G_{#1}}
\newcommand{\nkGA}[2]{G_{#1}^{(#2)}}

\newcommand{\resetall}{\setcounter{entry}{0}\setcounter{equation}{0}\setcounter{footnote}{0}}

\newcommand{\cbas}[1]{\gvec{\sigma}_{#1}}
\newcommand{\xbas}{\gvec{\sigma}_{1}}
\newcommand{\ybas}{\gvec{\sigma}_{2}}
\newcommand{\zbas}{\gvec{\sigma}_{3}}
\newcommand{\basis}[1]{\mvec{e}_{#1}}
\newcommand{\bbasis}[1]{\breve{\mvec{e}}_{#1}}
\newcommand{\dbasis}[1]{\mvec{e}^{#1}}
\newcommand{\bdbasis}[1]{\breve{\mvec{e}}^{#1}}
\newcommand{\nbasis}[1]{\hat{\mvec{e}}_{#1}}
\newcommand{\stabas}[1]{{\gamma}_{#1}}
\newcommand{\dstabas}[1]{{\gamma}^{#1}}
\newcommand{\ebas}[1]{e_{#1}}

\newcommand{\basisp}[1]{\mvec{e}'_{#1}}
\newcommand{\dbasisp}[1]{\mvec{e}'^{#1}}

\newcommand{\unitI}{\mathcal{I}}

\newcommand{\veps}{\gvec{\varepsilon}}
\newcommand{\bdum}{\breve{\_}}
\newcommand{\vecspace}{\mathcal{V}}
\newcommand{\tvecspace}{T_pM}
\newcommand{\vnorm}[1]{\lVert{#1}\rVert}
\newcommand{\dderiv}[1]{\boldsymbol{\partial}_{#1}}
\newcommand{\dderivd}[1]{\dot{\boldsymbol{\partial}}_{#1}}
\newcommand{\dderivl}[1]{\overleftarrow{\boldsymbol{\partial}}_{#1}}
\newcommand{\dderivlr}[1]{\overleftrightarrow{\boldsymbol{\partial}}_{#1}}
\newcommand{\dderivb}[2]{{#1}\cdot\boldsymbol{\partial}_{#2}}

\newcommand{\dmvd}[1]{{\partial}_{#1}}
\newcommand{\dmvdb}[2]{{#1}*{\partial}_{#2}}

\newcommand{\gsel}[2]{\langle{#1}\rangle_{#2}}
\newcommand{\gzsel}[1]{\langle {#1} \rangle}

%\newcommand{\rotm}[2]{{#1}{#2}\widetilde{#1}}
\newcommand{\rotm}[2]{{#1}{#2}{#1^{-1}}}

\newcommand{\uln}[1]{\underline{#1}}
\newcommand{\ulnb}[1]{\underline{\bar{#1}}}
\newcommand{\In}[1]{\mvec{I}_{#1}}

% For SR
\newcommand{\pfder}{\square}
\newcommand{\rfder}{\nabla}
\newcommand{\scder}{\mathfrak{D}}


\newcommand{\dprod}{\mathfrak{D}}


\DeclareMathOperator{\Pin}{Pin}
\DeclareMathOperator{\Spin}{Spin}

% Make the items smaller
\newcommand{\cramplist}{
	\setlength{\itemsep}{0in}
	\setlength{\partopsep}{0in}
	\setlength{\topsep}{0in}}
\newcommand{\cramp}{\setlength{\parskip}{.5\parskip}}
\newcommand{\zapspace}{\topsep=0pt\partopsep=0pt\itemsep=0pt\parskip=0pt}