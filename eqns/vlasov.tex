\chapter{The Vlasov Equation}

The Vlasov equations (or the Boltzmann equations) are the most
fundamental equation set in all of plasma physics. Unfortunately,
these equations are highly nonlinear and evolve in 6D phase-space,
making their direct use in most fusion problems basically
impossible. However, they serve as the starting point for deriving
other approximations that are more tractable.

\section{The Basic Equations}

Each species in a multi-component plasma is described by the Vlasov
equation for the evolution of a particle distribution function in 6D
phase-space. The particles move in electromagnetic fields that come
from two sources: (i) external coils and electrodes, and (ii) fields
generated by the motion of the particles themselves.  This
field-particle coupling makes the problem nonlinear: the fields tell
the particles how to move, and the particle motion itself modifies the
fields. Further, particle motion is also modified by collisions. With
the distribution function for species $s$, $f_s(\mvec{x},\mvec{v},t)$,
defined such that
$f_s(\mvec{x},\mvec{v},t)d\mvec{x}\thinspace d\mvec{v}$ is the number
of particles in a phase-space volume element
$d\mvec{x}\thinspace d\mvec{v}$, the Vlasov equation may be written as
\begin{align}
   \frac{\partial f_s}{\partial t}
   + \gcs \cdot (\mathbf{v}f_s)
   + \gvs \cdot (\mathbf{a}_s f_s)
   =
  \mathcal{H}_s,
  \label{eq:boltz}
\end{align}
where the acceleration of species $s$ is defined as
\begin{align}
    \mathbf{a}_s = \frac{q_s}{m_s} \left ( \mathbf{E} + \mathbf{v} \times \mathbf{B} \right ).
\end{align}
Here $\mvec{E}$ is the electric field, $\mvec{B}$ is the magnetic
field, $q_s$ and $m_s$ are the charge and mass of the plasma species,
and $\mathcal{H}_s$ are collision terms. We provide expressions for
collisions below. The electromagnetic fields are evolved with Maxwell
equations
\begin{align}
  \pfrac{\mvec{B}}{t} + \nabla\times\mvec{E} &= 0 \\
  \epsilon_0\mu_0\frac{\partial \mathbf{E}}{\partial t} -
  \nabla\times\mathbf{B} &= -\mu_0\sum_s \mathbf{J}_s \\
  \nabla \cdot \mvec{B} &= 0 \\
  \nabla \cdot \mvec{E} &= \frac{\varrho_c}{\epsilon_0}
\end{align}
The charge and current density are computed from
\begin{align}
  \varrho_c &= \sum_s q_s \int_{-\infty}^\infty f_s \thinspace d^3\mvec{v} \\
  \mvec{J}_s &= \int_{-\infty}^\infty \mvec{v} f_s \thinspace d^3\mvec{v}.
\end{align}

The Vlasov-Maxwell system is a formidable system of coupled, nonlinear
equations and describes vast physics that spans an enormous range of
temporal and spatial scales. Everything from electron oscillations to
slow, resistive evolution of objects in near equilibrium is contained
within this equation system. Directly solving the above set of
equations is possible: many production
particle-in-cell\cite{Bowers:2008} and continuum codes\cite{Juno:2018}
now exist that can evolve the distribution function of arbitrary
number of species. However, these kinetic solvers still remain
extremely expensive and impractical for many problems of interest in
fusion, laboratory and astrophysical plasmas.

\section{Conservation Properties}

We define the moment operator for any function $\varphi(\mvec{v})$ as
\begin{align}
  \langle \varphi(\mvec{v}) \rangle_s \equiv 
  \int_{-\infty}^{\infty} \varphi(\mvec{v}) f_s(t, \mvec{x}, \mvec{v}) \thinspace d^3\mvec{v}
\end{align}

We state the following propositions without proof. For proofs please
see, for example\cite{Juno:2018} or a good text-book on plasma
physics. In the following, $\Omega$ is the configuration-space domain
and $K$ the complete phase-space domain. Further, we assume that
$f_s(t, \mvec{x}, \mvec{v} \rightarrow\pm\infty) \rightarrow 0$ faster
than $\mvec{v}^n$ for finite $n$. We further assume that $\Omega$ is
\emph{periodic} and hence drop surface terms over $\partial
\Omega$. Alternately, one can assume that the distribution function
has compact support on the interior of $\Omega$ in which case also the
surface terms vanish.

\begin{proposition}
  The Vlasov-Maxwell system conserves particles:
  \begin{align}
    \frac{d}{dt} \int_\Omega \sum_s \langle 1 \rangle_s \thinspace
    d^3\mvec{x} = 0.
  \end{align}
\end{proposition}

\begin{proposition}
  The \emph{collisionless} Vlasov-Maxwell conserves the $L_2$ norm of
  the distribution function:
  that is
  \begin{align}
    \frac{d}{dt} \frac{1}{2} \int_K f_s^2 \thinspace
    d^6\mvec{z} = 0.
  \end{align}
\end{proposition}

\begin{proposition}
  The \emph{collisionless} Vlasov-Maxwell conserves the entropy $S = -
  f_s \ln f_s$ of the system:
  \begin{align}
    \frac{d}{dt} \int_K -f_s \ln f_s \thinspace
    d^6\mvec{z} = 0.
  \end{align}
\end{proposition}

\begin{remark}
  We remark that when collisions are included the $L_2$ norm of the
  distribution function and the entropy decay monotonically.
\end{remark}

\begin{proposition}
  The Vlasov-Maxwell system conserves total (particle plus field)
  momentum:
  \begin{align}
    \frac{d}{dt} \int_\Omega 
    \left(
    \sum_s \langle m_s \mvec{v} \rangle_s + \epsilon_0 \mvec{E}\times\mvec{B}
    \right) 
    d^3\mvec{x} = 0.
  \end{align}
\end{proposition}

\begin{proposition}
  The Vlasov-Maxwell system conserves total (particle plus field)
  energy:
  \begin{align}
    \frac{d}{dt} \int_\Omega 
    \left(
    \sum_s \langle \frac{1}{2} m_s \mvec{v}^2 \rangle_s + 
    \frac{\epsilon_0}{2} \mvec{E}^2 + \frac{1}{2\mu_0} \mvec{B}^2
    \right) 
    d^3\mvec{x} = 0.
  \end{align}
\end{proposition}

\begin{remark}
  The conservation laws above are \emph{global conservation laws},
  that is, involve the total (integrated over space)
  quantities. However, in general, \emph{local} conservation laws are
  far more powerful. The latter essentially state how quantities
  inside an arbitrary volume (and not just the whole space) change via
  fluxes through the surface bounding that volume.
\end{remark}

\section{Fokker-Planck Collision Operator}

The most complete collision operator is the \emph{Fokker-Planck
  operator} (FPO). Unfortunately, the FPO is notoriously complicated
and nonlinear and hence one often uses approximate collision
operators instead. In the following sections we list two such
operators, the Dougherty Lenard-Bernstein operator, and the
Bhatnagar-Gross-Krook operator.

In a multi-species plasma the FPO can be written as (see
NRL Plasma Formulary and the original Rosenbluth, MacDonald and Judd
(RMJ) paper\cite{Rosenbluth:1957}\footnote{Historically, the first
  derivation of the FPO in the case of Coulomb (inverse square law)
  was by Lev Landau in 1936\cite{Landau:1936}. Landau, however, writes
  the equation in an integral form and does not introduce the
  potentials as RMJ did. Curiously, the 1957 paper by RMJ does not
  mention Landau's work at all. The original papers remain highly
  readable and still provide the best derivations of the equations.})
\begin{align}
  \mathcal{H}_s = - \gvs\cdot
  \big[
  \mvec{a}_s f_s - \frac{1}{2} \gvs\cdot\left( \mvec{D}_s f_s \right)
  \big]
   \label{eq:fpo}
\end{align}
where recall that $\gvs$ operator is the gradient operator in
\emph{velocity space}. The drag velocity and diffusion tensor are
given by
\begin{align}
  \mvec{a}_s &= \gvs h_s \\
  \mvec{D}_s &= \gvs\otimes\gvs g_s
\end{align}
where
\begin{align}
  h_s &= \sum_b \Gamma_{sb} \bigg( 1 + \frac{m_s}{m_b} \bigg) H_b  \\
  g_s &= \sum_b \Gamma_{sb} G_b.
\end{align}
are scalar potentials. Here
\begin{align}
  \Gamma_{ab} = 4\pi\lambda_{ab}\frac{q_a^2 q_b^2}{m_a^2}
\end{align}
where $\lambda_{ab}$ is the Coulomb logarithm, and $q_a$ and $m_a$ are
the charge and mass of the species respectively.  The potentials $H_s$
and $G_s$ are the Rosenbluth potentials and are determined from
\begin{align}
  H_s(\mvec{v}) &=  \int \frac{f_s(\mvec{v}')}{ |\mvec{v}-\mvec{v}'|} \thinspace d^3\mvec{v}',  \label{eq:Ha} \\
  G_s(\mvec{v}) &= \int f_s(\mvec{v}') | \mvec{v}-\mvec{v}'| \thinspace d^3\mvec{v}'. \label{eq:Ga}
\end{align}
Using the identity
\begin{align}
  \gvs^2|\mvec{v}-\mvec{v}'|^{-1}
  = -4\pi\delta^3(\mvec{v}-\mvec{v}')
\end{align}
and as $\gvs^2 |\mvec{v}| = 2/|\mvec{v}|$ we can derive alternate
expressions for the Rosenbluth potentials as
\begin{align}
  \gvs^2 H_s &= -4\pi f_s \\
  \gvs^2 G_s &= 2 H_s.
\end{align}

As is clear from these equations, the FPO is a complicated 3D (in
velocity space) nonlinear integro-differential equation, coupling all
species via the Rosenbluth potentials. Its solutions display rich
structure, specially when combined with the particle motion in
self-consistent electromagnetic fields. Designing a general production
solver for the case of multi-species FPO poses a formidable challenge.

I should mention that this is not the \emph{exact} form of the FPO as
given in the RMJ paper. There they assume that all species have the
\emph{same} absolute value of charge $|e|$, allowing them to write
$\Gamma_a = \Gamma_{aa}$. With this assumption a little algebra shows
that the equations listed above are indeed identical to the ones in
the RMJ paper.

We can easily derive the following two relations:
\begin{align}
  \Tr(\mvec{D}_s) = \gvs^2 g_s = 2\sum_b \Gamma_{sb} H_b
\end{align}
and\footnote{We have
    $\breve{\nabla}\cdot(\nabla\otimes\breve{\nabla} g_s) 
    = \Tr(\breve{\nabla}\otimes\nabla\otimes\breve{\nabla} g_s)
    = \nabla\otimes \Tr(\nabla\otimes\nabla g_s)
    = \nabla \nabla^2 g_s
    $.
}
\begin{align}
  \gvs\cdot\mvec{D}_s =
  \gvs\cdot(\gvs\otimes\gvs g_s)
  = \gvs \gvs^2 g_s
  = 2 \sum_b \Gamma_{sb} \gvs H_b.
\end{align}
As
$\gvs\cdot\left( \mvec{D}_s f_s \right) = f_s \gvs\cdot \mvec{D}_s
+ \gvs f_s\cdot \mvec{D}_s$ the latter expression allows us to write
the FPO as
\begin{align}
  \mathcal{H}_s = - \frac{1}{2} \gvs\cdot
  \big[
  \underbrace{\mvec{a}'_s f_s}_{\textrm{drag}}
  -
  \underbrace{\gvs f_s \cdot\mvec{D}_s}_{\textrm{diffusion}}
  \big]
  \label{eq:fpo-2}
\end{align}
where
\begin{align}
  \frac{1}{2}\mvec{a}'_s = \mvec{a}_s - \sum_b \Gamma_{sb} \gvs H_b
  = \sum_b \Gamma_{sb} \frac{m_s}{m_b} \gvs H_b.
\end{align}
Written in the form \eqr{\ref{eq:fpo-2}} the FPO clearly has two
competing terms: the first \emph{drag} term and the second the
\emph{diffusion} term. For the special case of a single species we
have $h_s = 2\Gamma_s H_s$ and rather elegantly,
$\mvec{a}'_s = \mvec{a}_s$.

\section{Dougherty Lenard-Bernstein Operator}

A simplified collision operator that is often used is the Dougherty
Lenard-Bernstein operator (DLBO). This has the same structure as the
FPO, \eqr{\ref{eq:fpo}}, except that the drag velocity and diffusion
tensor are determined from
\begin{align}
  \mvec{a}_{s} &= -\nu_{ss} (\mvec{v}-\mvec{u}_s) 
                 - \sum_{r\ne s} \nu_{sr} (\mvec{v}-\mvec{u}_{sr})  \label{eq:dlbo-a} \\
  \mvec{D}_{s} &= 2 \nu_{ss}v_{th,s}^2 \mvec{g}_{\mvec{v}}  +
                 \sum_{r\ne s} 2 \nu_{sr}v_{th,sr}^2 \mvec{g}_{\mvec{v}} \label{eq:dlbo-D}
\end{align}
where $\mvec{g}_{\mvec{v}}$ is the metric tensor in velocity
space. Further, $\nu_{sr}$ is the collision frequency between
particles of species $s$ and $r$, and $\mvec{u}_s$ and
$v_{th,s} = \sqrt{T_s/m_s}$ are the mean (drift) velocity and thermal
speed of particles of species $s$ respectively, and the quantities
with subscript $sr$ are intermediate quantities that need to be
determined (partly) from momentum and energy conservation, that is
chosen to satisfy \eqr{\ref{eq:mom-constraint}} and
\eqr{\ref{eq:energy-constraint}}. See\cite{Francisquez:2022} for
details on how the intermediate quantities can be determined.

\section{Conservation Properties and Constraints For FPO-Type
  Collision Operators}

The FPO conserves the particle count, total momentum and total
energy. Starting from \eqr{\ref{eq:fpo}} we can derive the constraints
to ensure momentum and energy by taking $m_s\mvec{v}$ and $m_s v^2/2$
moments to arrive at the following constraint
\begin{align}
  \sum_s \int_{-\infty}^\infty m_s \mvec{a}_s f_s \thinspace
  d^3\mvec{v} = 0 \label{eq:mom-constraint}
\end{align}
for momentum conservation, and
\begin{align}
  \sum_s \int_{-\infty}^\infty m_s
  \left[
  \mvec{v}\cdot\mvec{a}_s + \frac{1}{2}\Tr(\mvec{D}_s)
  \right]
  f_s \thinspace d^3\mvec{v} = 0 \label{eq:energy-constraint}
\end{align}
for energy conservation.

\begin{remark}
  These constraints depend only on the form of \eqr{\ref{eq:fpo}} and
  not on the details of how $\mvec{a}$ and $\mvec{D}$ are
  computed. Hence, we must show that for the \emph{specific case} of
  FPO in which the drag and diffusion terms are computed from
  Rosenbluth potentials, they are indeed satisfied. If we instead
  choose to use the DLBO then we must also ensure that these
  constraints are satisfied, but now for the expressions
  Eqns.\thinspace(\ref{eq:dlbo-a}) and (\ref{eq:dlbo-D}).
\end{remark}

\section{Bhatnagar-Gross-Krook Collision Operator}

An even simpler collision operator is the \emph{Bhatnagar-Gross-Krook}
(BGK) collision operator. This operator relaxes the distribution
function to a local Maxwellian for that species. We can write this as
$\mathcal{H}_s = \mathcal{H}_{ss} + \sum_{r\ne s} \mathcal{H}_{sr}$,
where
\begin{align}
  \mathcal{H}_{ss} = 
    \nu_{ss}
    \big(
    f_{Ms} - f_s
    \big); \quad
  \mathcal{H}_{sr} = 
    \nu_{sr}
    \big(
    f_{Msr} - f_s
    \big)
\end{align}
where $\nu_{sr}$ is the collision-frequency of species $s$ with
species $r$ and $f_{Ms}$, $f_{Msr}$ are a Maxwellians given by
\begin{align}
  &f_{Ms} = n_s 
    \left(
    \frac{m_s}{2\pi T_s}
    \right)^{3/2}
    e^{-m_s (\mvec{v}-\mvec{u}_s)^2/ 2T_s} \\
  &f_{Msr} = n_s 
    \left(
    \frac{m_s}{2\pi T_{sr}}
    \right)^{3/2}
    e^{-m_s (\mvec{v}-\mvec{u}_{sr})^2/ 2T_{sr}}
\end{align}
and where $\mvec{u}_{sr}$ and $T_{sr}$ are intermediate velocities and
temperatures. Greene\cite{Greene:1973} provides an approach to
constructing these quantities based on conservation of momentum and
energy. We note that the description of BGK operator in the NRL Plasma
Formulary is misleading as it seems to indicate that the operator is
not conservative. It is hence best to refer to the original
Greene\cite{Greene:1973} paper.
