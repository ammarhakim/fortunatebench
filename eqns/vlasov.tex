\chapter{The Vlasov Equation}

The Vlasov equations (or the Boltzmann equations) are the most
fundamental equation set in all of plasma physics. Unfortunately,
these equations are highly nonlinear and evolve in 6D phase-space,
making their direct use in most fusion problems basically
impossible. However, they serve as the starting point for deriving
other approximations that are more tractable.

\section{The Basic Equations}

Each species in a multi-component plasma is described by the Vlasov
equation for the evolution of a particle distribution function in 6D
phase-space. The particles move in electromagnetic fields that come
from two sources: (i) external coils and electrodes, and (ii) fields
generated by the motion of the particles themselves.  This
field-particle coupling makes the problem nonlinear: the fields tell
the particles how to move, and the particle motion itself modifies the
fields. Further, particle motion is also modified by collisions. With
the distribution function for species $s$, $f_s(\mvec{x},\mvec{v},t)$,
defined such that
$f_s(\mvec{x},\mvec{v},t)d\mvec{x}\thinspace d\mvec{v}$ is the number
of particles in a phase-space volume element
$d\mvec{x}\thinspace d\mvec{v}$, the Vlasov equation may be written as
\begin{align}
   \frac{\partial f_s}{\partial t}
   + \gcs \cdot (\mathbf{v}f_s)
   + \gvs \cdot (\mathbf{a}_s f_s)
   =
  \mathcal{H}^s,
  \label{eq:boltz}
\end{align}
where the acceleration of species $s$ is defined as
\begin{align}
    \mathbf{a}_s = \frac{q_s}{m_s} \left ( \mathbf{E} + \mathbf{v} \times \mathbf{B} \right ).
\end{align}
Here $\mvec{E}$ is the electric field, $\mvec{B}$ is the magnetic
field, $q_s$ and $m_s$ are the charge and mass of the plasma species,
and $\mathcal{H}^s$ are collision terms. We provide expressions for
collisions below. The electromagnetic fields are evolved with Maxwell
equations
\begin{align}
  \pfrac{\mvec{B}}{t} + \nabla\times\mvec{E} &= 0 \\
  \epsilon_0\mu_0\frac{\partial \mathbf{E}}{\partial t} -
  \nabla\times\mathbf{B} &= -\mu_0\sum_s \mathbf{J}_s \\
  \nabla \cdot \mvec{B} &= 0 \\
  \nabla \cdot \mvec{E} &= \frac{\varrho_c}{\epsilon_0}
\end{align}
The charge and current density are computed from
\begin{align}
  \varrho_c &= \sum_s q_s \int_{-\infty}^\infty f_s \thinspace d^3\mvec{v} \\
  \mvec{J}_s &= \int_{-\infty}^\infty \mvec{v} f_s \thinspace d^3\mvec{v}.
\end{align}

The Vlasov-Maxwell system is a formidable system of coupled, nonlinear
equations and describes vast physics that spans an enormous range of
temporal and spatial scales. Everything from electron oscillations to
slow, resistive evolution of objects in near equilibrium is contained
within this equation system. Directly solving the above set of
equations is possible: many production
particle-in-cell\cite{Bowers:2008} and continuum codes\cite{Juno:2018}
now exist that can evolve the distribution function of arbitrary
number of species. However, these kinetic solvers still remain
extremely expensive and impractical for many problems of interest in
fusion, laboratory and astrophysical plasmas.

\section{Conservation Properties}

We define the moment operator for any function $\varphi(\mvec{v})$ as
\begin{align}
  \langle \varphi(\mvec{v}) \rangle_s \equiv 
  \int_{-\infty}^{\infty} \varphi(\mvec{v}) f_s(t, \mvec{x}, \mvec{v}) \thinspace d^3\mvec{v}
\end{align}

We state the following propositions without proof. For proofs please
see, for example\cite{Juno:2018} or a good text-book on plasma
physics. In the following, $\Omega$ is the configuration-space domain
and $K$ the complete phase-space domain. Further, we assume that
$f_s(t, \mvec{x}, \mvec{v} \rightarrow\pm\infty) \rightarrow 0$ faster
than $\mvec{v}^n$ for finite $n$. We further assume that $\Omega$ is
\emph{periodic} and hence drop surface terms over $\partial
\Omega$. Alternately, one can assume that the distribution function
has compact support on the interior of $\Omega$ in which case also the
surface terms vanish.

\begin{proposition}
  The Vlasov-Maxwell system conserves particles:
  \begin{align}
    \frac{d}{dt} \int_\Omega \sum_s \langle 1 \rangle_s \thinspace
    d^3\mvec{x} = 0.
  \end{align}
\end{proposition}

\begin{proposition}
  The \emph{collisionless} Vlasov-Maxwell conserves the $L_2$ norm of
  the distribution function:
  that is
  \begin{align}
    \frac{d}{dt} \frac{1}{2} \int_K f_s^2 \thinspace
    d^6\mvec{z} = 0.
  \end{align}
\end{proposition}

\begin{proposition}
  The \emph{collisionless} Vlasov-Maxwell conserves the entropy $S = -
  f_s \ln f_s$ of the system:
  \begin{align}
    \frac{d}{dt} \int_K -f_s \ln f_s \thinspace
    d^6\mvec{z} = 0.
  \end{align}
\end{proposition}

\begin{remark}
  We remark that when collisions are included the $L_2$ norm of the
  distribution function and the entropy decay monotonically.
\end{remark}